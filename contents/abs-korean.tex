
\abstractKr{

원격 협업 기술은 작업자들이 물리적으로 한곳에 모이지 않고도 협업할 수 있게 만들어 주었으며, 작업자들은 협업하기 위해 먼 거리를 이동하지 않아도 되므로 시간 및 비용을 절약할 수 있게 되었다. 근래의 원격 협업은 단순한 화상회의를 넘어 가상현실(\acrshort{vr})과 증강현실(\acrshort{ar})을 활용하는 확장현실(\acrshort{xr}) 기술을 통해 이루어지고 있다. \acrshort{ar}을 이용한 원격 협업은 현장 작업자의 카메라 뷰가 공유되고 원격지의 작업자가 음성 또는 드로잉 등을 통해 작업 지시를 전달하는 방식으로 이루어진다. 이때, 원격 작업자는 현장을 관찰하는 시점을 자유롭게 바꾸지 못하고 현장 작업자의 카메라 움직임에 의존하여 수동적으로 현장의 상황을 파악하게 되어 협업이 더디게 이루어진다. 다른 방식의 \acrshort{xr} 협업으로, 원격 작업자가 좀 더 몰입적인 환경에서 현장을 체험할 수 있도록 현장을 3차원 재구성한 디지털 트윈 공간에서 협업을 진행하는 방법이 있다. 하지만, 현장에 많은 센서를 설치하거나 또는, 현장을 본뜬 3D 모델을 만드는 등 부가적인 작업이 많아 이 방법 또한 실용적이지 못하였다.

본 연구의 `실시간 디지털 트윈 공간 구성을 이용한 웹 기반 원격 \acrshort{xr} 협업 방법'은 협업 대상과 배경 공간을 구분하여, 협업 대상만 사전에 3D 모델링을 하고 배경은 협업을 할 때 현장에서 실시간으로 재구성하는 방식의 협업을 제안한다. 이 방법은 협업을 위한 3D 모델링 작업을 최소화하고 센서 설치를 필요로 하지 않아 기존 방법보다 실용적인 \acrshort{xr} 협업 경험을 제공한다. 본 연구는 두 개의 하위 기술로 나뉜다. `웹 기반의 원격 \acrshort{xr} 협업'은 이미지 트래킹 기술을 이용해 협업 대상의 위치를 추적하고 추적한 위치 정보를 원격 작업자에게 동기화한다. 이때, 획득한 물체의 위치는 카메라 좌표계에서 상에서 정의되는데 이를 원격 작업자가 사용하는 월드 좌표계로 변환하기 위해 `스페이스 마커'를 도입하였다. 그리고, 원격 작업자와 현장 작업자가 각각 이용하는 \acrshort{vr} 콘텐츠와 \acrshort{ar} 콘텐츠를 한 번에 표현할 수 있는 \acrshort{xr} 콘텐츠 표현을 정의하여 동일한 협업 시나리오에 대해 중복적인 콘텐츠를 생성하지 않아도 되도록 만들었다. `실시간 디지털 트윈 공간 구성'은 \acrshort{lidar} 센서를 이용하여 현장의 깊이 데이터를 수집하고 핀홀 카메라 모델을 통해 포인트 클라우드 데이터로 3차원 재구성한다. 재구성한 포인트 클라우드 데이터에서 협업 대상에 해당하는 부분은 마스킹하여 제거하고 웹 표준의 \acrshort{p2p} 통신 기술인 \acrshort{webrtc}를 이용하여 원격 작업자에게 전달한다. 원격 작업자는 전달받은 포인트 클라우드 데이터를 누적하여 현장을 복제한 디지털 트윈 공간을 구성하고, 이를 통해 더욱 직관적으로 현장의 상황을 파악할 수 있게 된다. 현장 작업자는 단순한 드로잉이 아닌 3D 모델을 협업 대상 위에 겹쳐 봄으로써 직관적으로 작업 정보를 파악할 수 있어 협업의 능률을 향상시킬 수 있다. 또한, 본 연구에서 제안하는 \acrshort{xr} 협업 방법은 사람 간의 협업뿐만 아니라 원격 로봇 조종 등과 같은 다양한 미래 산업 분야에도 응용될 수 있을 것으로 본다.

}
