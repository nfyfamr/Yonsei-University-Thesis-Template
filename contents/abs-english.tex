
\abstractEng{

Remote collaboration technology has enabled workers to work together without physically gathering, saving time and money by not having to travel long distances to collaborate. Recent remote collaboration has been achieved through \acrfull{xr} that leverage \acrfull{vr} and \acrfull{ar}, beyond simple video conferencing. Remote collaboration using \acrshort{ar} takes place in such a way that local workers share their camera views and that remote workers communicate task instructions via voice or drawing on screen, etc. At this time, the remote worker cannot freely change the point of observing the site and relies on the camera movement of the local worker to grasp the situation of the site, which slows collaboration. In the other method of using \acrshort{xr}, there is a method to proceed collaboration in a three-dimensional reconstructed digital twin workspace where remote workers experience the local site in a more immersive environment. However, this method has also been impractical due to many additional tasks, such as installing many sensors in the local site, or creating 3D models replicating the local site.

In this work, we propose the method of `Web-based Remote \acrshort{xr} Collaboration Using Real-time Digital Twinning of Workspace' distinguishing collaboration targets and background spaces, in which only collaboration targets are 3D modeled in advance, and backgrounds are reconstructed in real-time. The method minimizes the 3D modeling task for collaboration and does not require sensor installation, providing a more practical \acrshort{xr} collaboration experience than previous methods. This study is divided into two sub-techniques. The `web-based remote \acrshort{xr} collaboration' uses image tracking techniques to track the pose of collaboration targets and synchronize the pose information to remote workers. Because the pose of tracked object is defined on the camera coordinate system, we introduced a `space marker' to transform it into a world coordinate system used by remote workers. In addition, we defined a \acrshort{xr} content representation method that can represent \acrshort{vr} content used by remote workers and \acrshort{ar} content used by local workers at once, eliminating the need to generate duplicate content for the same collaborative scenario. The `real-time digital twinning of workspace' technique collects depth data of the local site using a \acrshort{lidar} sensor and reconstructs them as point cloud data via pinhole camera models. Then, the parts corresponding to the collaboration target in the point cloud data are removed and sent to remote workers using \acrshort{webrtc}, a web standard communication technology. Remote workers construct digital twin spaces that replicate the local site by accumulating the delivered point cloud data, which allows them to more intuitively grasp the situation of the local site actually they did not visit. Also, local workers can intuitively grasp task instruction by watching the overlapped 3D models over collaborative targets rather than simple drawing on screen, which can improve collaboration efficiency. Furthermore, we believe that our proposed \acrshort{xr} collaboration method can be applied not only to human-to-human collaboration, but also to various industrial fields such as remote robot control.

}
